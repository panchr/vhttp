\section{Conclusions}
\label{sec:conclusion}

In this paper, we introduced vHTTP, a locally-run HTTP proxy that provides
mitigation against localized Man-in-the-Middle attacks. It does so through the
concept of vantage points, which are nodes, ideally geographically-diverse,
that provide perspectives onto the actual content of a response. Our proxy
uses normal HTTP proxies, of which there are many publicly available, as
vantage points because of their ease of use and access. vHTTP does not require
any additional infrastructure or modifications to server-side software and
instead, works solely on clients. Thus, it can easily be used by
security-conscious users who are concerned about the validity of an HTTP
response. We believe it provides an adequate technique to avoid local MITM
attacks when HTTPS is not available.

\subsection{Future Work}

Although vHTTP is functional currently, there are numerous avenues with which
to take the project. First, different hash functions can be explored to provide
either different levels of security or flexibility in what component of each
response is desired to be verified across vantage points. In addition,
it may also be useful to support configuration as to which response is chosen
out of a consensus set. Currently, vHTTP arbitrarily picks a response out of a
consensus set. Although this works under the assumption that the responses are
essentially equivalent except for minor details (such as the \texttt{Date}
header, which can be slightly off due to response timing), different metrics
can be considered.

Two final pieces of future work concern vantage points. It may be worth
considering using Tor exit nodes, similar to DoubleCheck, as vantage points.
This allows for easier discovery and configuration, as well being more secure
than using HTTP proxies. Alternatively, adding automatic discovery of vantage
points would also be beneficial so that vHTTP can be even easier to use.
