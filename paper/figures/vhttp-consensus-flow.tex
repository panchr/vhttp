\begin{figure*}
  \centering
  \begin{adjustbox}{width=\textwidth}
  \begin{tikzpicture}[auto]
    \tikzset{every node}=[font=\tiny\sffamily]

    \node [block] at (0, 0) (client) {Client};
    \node [block] at (2, 0) (vhttp) {vHTTP};
    \node [block] at (4, 2) (vp-A) {Vantage Point A};
    \node [block] at (4, 1) (vp-B) {Vantage Point B};
    \node [block] at (4, 0) (vp-C) {Vantage Point C};
    \node [block] at (4, -1) (vp-D) {Vantage Point D};
    \node [block] at (4, -2) (vp-E) {Vantage Point E};
    \node [block] at (6, 0) (vhttp-rec) {vHTTP};
    \node [block] at (8, 0) (client-rec) {Client};

    \path[->, every
        node/.style={font=\sffamily\tiny,midway,anchor=center,fill=white}]
      (client) edge node {$R$} (vhttp)
      (vhttp) edge node {$R'$} (vp-A)
              edge node {$R'$} (vp-B)
              edge[dashed] node {$R'$} (vp-C)
              edge node {$R'$} (vp-D)
              edge node {$R'$} (vp-E)
      (vp-A) edge[ForestGreen] node {$S_A$} (vhttp-rec)
      (vp-B) edge[ForestGreen] node {$S_B$} (vhttp-rec)
      % (vp-C) edge[dashed] node {$S_C$} (vhttp-rec)
      (vp-D) edge[red] node {$S_D$} (vhttp-rec)
      (vp-E) edge[ForestGreen] node {$S_E$} (vhttp-rec)
      (vhttp-rec) edge[ForestGreen] node {$S_E$} (client-rec);

  \end{tikzpicture}
  \end{adjustbox}
  \caption{
    \emph{Flow of data when vHTTP achieves consensus.}
    In this example, the client proxies a request $R$ through vHTTP. vHTTP
    sends $R$, with some minor modifications\protect\footnotemark,
    along to the five vantage points configured. Vantage points $A$, $B$, and
    $C$ respond. Assuming a default threshold of $0.5$, they achieve
    consensus, so one of those responses (in this case, we arbitrarily chose
    $S_E$) is sent back to the user. $D$ is either a malicious proxy or a MITM
    attack is being performed on this path, so its response is discarded as
    against the largest consensus, and $C$ never responds because it is
    unavailable. Note that there is only one vHTTP instance and one client;
    they are separated (on the receiving end) to declutter the diagram.
    }
  \label{fig:vhttp-consensus-flow}
\end{figure*}

\footnotetext{
  The primary modification is to ensure that the
  request has the \texttt{Connection} header set to \texttt{close} so that
  the connection does not remain open
  }
