\section{Related Work}
\label{sec:related}

Vantage point-based systems have been designed in the past and have influenced
the design of vHTTP. One such system is Perspectives, which uses a system of
notary servers that track the public keys of network services
\cite{Perspectives}. Clients presented with an untrusted public key for a
service can query different notary servers and based on those servers'
perspective of what the key should be, accept or reject the key. This is
similar to vHTTP in that the vHTTP proxy will forward the client's request to
different proxies and determine if consensus exists among those proxies.

However, vHTTP differs from Perspectives in that it focuses on HTTP and not
SSH keys, which results in dealing with different data. vHTTP must determine
consensus not based on the entire response, but rather, on a subset of the
response that is related to the response's content. Perspectives, on the other
hand, can simply base consensus off of the entirety of the public key. In
addition, vHTTP does not require new external infrastructure. vHTTP operates
with existing HTTP proxies as vantage points; setting up so-called notary
servers is not needed. Thus, vHTTP can be used by clients without additional
support, whereas Perspectives requires additional infrastructure to be setup
first.

Another similar system is DoubleCheck, which performs the same task as
Perspectives – validating public keys or self-signed certificates when
connecting to servers over SSH or HTTPS \cite{DoubleCheck}. However, it does
so using Tor, the distributed and anonymous routing protocol and network. This
is easier to get started with than Perspectives because it utilizes existing
infrastructure and more importantly, relies on an established protocol.
However, DoubleCheck relies on a single Tor exit node as a vantage point.
vHTTP has the advantage of running the request across multiple vantage points
concurrently by default.

We note that neither Perspectives nor DoubleCheck is oriented towards HTTP. In
addition, they face the same problem that vHTTP does: if all of the vantage
points are compromised \emph{or} if the destination server is compromised, the
vantage points cannot help in verifying the authenticity and integrity of the
content at hand.
