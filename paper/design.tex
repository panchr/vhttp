\section{Design}
\label{sec:design}

vHTTP is implemented as an HTTP proxy that can be run locally by security-aware
people. Users can configure their clients (such as \texttt{curl},
\texttt{wget}, or even Python's \texttt{requests}) to use vHTTP as a HTTP
proxy. All HTTP requests from this client are then sent to vHTTP instead of
to the actual domain. vHTTP performs a MITM mitigation technique on behalf of
the client, and then responds with the appropriate data if the mitigation
succeeds.

Because vHTTP is written an HTTP proxy, and not an extension to any specific
client itself, it can be used by a wide-variety of clients. As long as a client
supports using HTTP proxies, which most do, vHTTP can be used without hassle.

\subsection{Vantage Points}

vHTTP performs MITM mitigation by using \textit{vantage points}. Vantage points
are different locations from which the request is performed. In other words,
vantage points provide different perspectives on the result of a request. If
vantage points are geographically-diverse, or located around the world, then
performing a MITM on such content is a much more difficult task, as each
individual vantage point must be compromised.

The entire idea behind vHTTP is to utilize the general property that MITM
attacks tend to be localized. A malicious agent can use DNS cache poisoning,
BGP route hijacking, or act as a middle box on a specific route or subset of
routes to a destination. However, unless performing a massive-scale attack, it
is infeasible and unlikely that an agent can attack all of the routes to a
destination. Agents who could feasibly perform a MITM attack on every route to
a destination are likely targeting important content - at which point the user
should not be accessing such content over HTTP, anyway. vHTTP is not designed
to mitigate large-scale attacks but instead, work around localized ones.

vHTTP is designed to be used with minimal overhead by the user. The user should
be able to start vHTTP, configure their client to use vHTTP as an HTTP proxy,
and then use their client. They should \emph{not} have to figure out how to set
up different vantage points across the globe. vHTTP is able to use
geographically-diverse vantage points, without the user having to set these up,
by using existing HTTP proxies. The user provides a list of proxies to use and
vHTTP performs the user's request through all of the provided proxies, as shown
in Figure \ref{fig:proxy-data-flow}. There are many free HTTP proxies around
the world, and so finding such a list is not difficult.

\begin{figure}[b]
  \centering

  \begin{tikzpicture}[node distance = 2cm, auto]
    \node [block] (client) {Client};
    \node [block, left of=client] (vhttp) {vHTTP};
    \node [block, left of=vhttp] (http-proxy) {HTTP Proxy};

    \draw[->, to path={-- (\tikztotarget)}]
      (client) edge (vhttp)
      (vhttp) edge (client)
      (vhttp) edge (http-proxy)
      (http-proxy) edge (vhttp);

  \end{tikzpicture}

  \caption{Flow of data in vHTTP.}
  \label{fig:proxy-data-flow}
\end{figure}

\subsection{Consensus}

Each request routed through another HTTP proxy is stored until all of the
proxies have either responded or failed. Then, the $N$ successful responses are
checked for a \textit{consensus}. Consensus occurs when at least a
user-specified threshold, $T$, of the responses agree. Responses $R_i$ and
$R_j$ agree if $H(R_i) = H(R_j)$ where $H$ is a hash function of our choosing.
