\section{Design}
\label{sec:design}

vHTTP is implemented as an HTTP proxy that can be run locally by security-aware
people. Users can configure their clients (such as \texttt{curl},
\texttt{wget}, or even Python's \texttt{requests}) to use vHTTP as a HTTP
proxy. All HTTP requests from this client are then sent to vHTTP instead of
to the actual domain. vHTTP performs a MITM mitigation technique on behalf of
the client, and then responds with the appropriate data if the mitigation
succeeds.

Because vHTTP is written as an HTTP proxy, and not an extension to any specific
client itself, it can be used by a wide-variety of clients. As long as a client
supports using HTTP proxies, which most do, vHTTP can be used without hassle.
Designing vHTTP as a regular proxy has this advantange of client flexibility,
but it comes at the cost of added overhead. Instead of the MITM mitigation
happening on the client side, it occurs on this additional vHTTP proxy; this,
of course, requires the additional step of requests going through vHTTP. In
addition, vHTTP's mitigation only works if the client's connection to vHTTP
is secure. As such, vHTTP is designed to be simple enough that users can run
it on their own hosts; this lets vHTTP act as a local HTTP proxy. Running it
locally has the added benefit of minimizing some of the network overhead. There
is more discussion about the efficiency and overhead present in the evaluation,
along with benchmarks of vHTTP's performance.

\subsection{Vantage Points}

vHTTP performs MITM mitigation by using \textit{vantage points}. Vantage points
are different locations from which the request is performed. In other words,
vantage points provide different perspectives on the result of a request. If
vantage points are geographically-diverse, or located around the world, then
performing a MITM on such content is a much more difficult task, as each
individual vantage point must be compromised. The concept of vantage points
is utilized in various applications already; it was proposed as a method to
avoid blindly trusting the public key on first use in SSH \cite{Perspectives}
or to prevent attacks on DNS \cite{Hotpets17, ConfiDNS}.

The entire idea behind vHTTP is to utilize the general property that MITM
attacks tend to be localized \cite{MITM-Survey}. A malicious agent can use DNS
cache poisoning, BGP AS-path poisoning \cite{Hotpets17}, or act as a middle
box on a specific route or subset of routes to a destination. However, unless
performing a massive-scale attack, it is infeasible and unlikely that an agent
can attack all of the routes to a destination. Agents who could feasibly
perform a MITM attack on every route to a destination are likely targeting
important content - at which point the user should not be accessing such
content over HTTP, anyway. vHTTP is not designed to mitigate large-scale
attacks but instead, work around localized ones.

vHTTP is designed to be used with minimal overhead by the user. The user should
be able to start vHTTP, configure their client to use vHTTP as an HTTP proxy,
and then use their client. They should \emph{not} have to figure out how to set
up different vantage points across the globe. vHTTP is able to use
geographically-diverse vantage points, without the user having to set these up,
by using existing HTTP proxies. The user provides a list of proxies to use and
vHTTP performs the user's request through all of the provided proxies, as shown
in Figure \ref{fig:proxy-data-flow}. There are many free HTTP proxies around
the world, and so finding such a list is not difficult
\cite{HideMyName-Proxies, NordVPN-Proxies, FreeProxyList}. In addition to using
these proxies, vHTTP also uses an additional vantage point: the original
intended server.\footnote{Like many of vHTTP's features, this can be disabled;
vHTTP is highly configurable to the user's needs.} This is done to provide an
additional perspective.

We refer to vantage points and HTTP proxies interchangeably in this paper, as
they are synonymous for our purposes. Although we desire that vantage points
have this feature of geographic diversity, vHTTP cannot easily enforce that
and does not attempt to. Instead, we hope that users will use vantage points
that do share this feature so that they can obtain a greater confidence in the
shared response.

\begin{figure}[b]
  \centering

  \begin{tikzpicture}[node distance = 2cm, auto]
    \tikzset{every node}=[font=\small\sffamily]
    \node [block] (client) {Client};
    \node [block, right of=client] (vhttp) {vHTTP};
    \node [block, right of=vhttp] (http-proxy) {HTTP Proxy};

    \draw[->, to path={-- (\tikztotarget)}]
      (client) edge (vhttp)
      (vhttp) edge (client)
      (vhttp) edge (http-proxy)
      (http-proxy) edge (vhttp);

  \end{tikzpicture}

  \caption{Flow of data in vHTTP for a single proxy.}
  \label{fig:proxy-data-flow}
\end{figure}

\subsection{Response Consensus}

Each request routed through a vantage point is stored until all of the
vantage points have either responded or failed. Then, the $N$ responses are
checked for a \textit{consensus} \cite{Perspectives}. A successful consensus
occurs when at least a user-specified threshold, $T$, of the responses agree.
Responses $R_i$ and $R_j$ agree if $H(R_i) = H(R_j)$ where $H$ is a hash
function of our choosing. This agreement can be extended to any $n$ responses;
that is, responses $R_1, R_2, ..., R_n$ all agree if
$H(R_i) = H(R_j) \forall i, j \in \{1...n\}$.

Thus, a successful consensus is reached if a subset $C$, of size at least $T$,
of the responses can be found such that all of the responses in $C$ are in
agreement. We call the subset $C$ that has successful consensus a
\textit{consensus set}. Note that because the user chooses $T$, they can vary
$T$ depending on how strict of a guarantee they want on the response.
Practically, the threshold is inputted as a fraction of $N$; setting it to 0
provides no added security and setting it to 1 requires all vantage points to
agree. For an example of consensus, see Figure \ref{fig:vhttp-consensus-flow}.

We choose the hash function $H$ based on this one key property: $H(R)$ for a
response $R$ should be representative of the \emph{content} in $R$. Because
vHTTP is concerned with the finding the accurate content for a request, the
basis for consensus should take the content of each response into account.
$H$ is thus defined as

\begin{align*}
  H(R) = sha256(R.status + R.body + R.headers)
\end{align*}

where $+$ is the concatenation operator. $R.status$ is the HTTP status code of
$R$, $R.body$ is the body of $R$, and $R.headers$ is a string representation
of certain HTTP headers and their values. The headers used by default are
\texttt{Content-Type}, \texttt{Content-Language}, and \texttt{ETag}. These
headers are included in $H(R)$ because they are directly related to the
content. In addition, the \texttt{Host} header is included because it is
unlikely to vary across requests. This is all run through the standard $sha256$
hash function so that $H$ maintains common properties of good hash functions
(different values have different hashes, equivalent values have the same hash,
collision probability is low, and so forth). If $R$ is a response that failed
or did not complete, $H(R) = \texttt{null}$.

From this, we get the property that all consensus sets are necessarily
disjoint. A response $R$ has a single hash value and so it belongs to a single
consensus set; $C$ with a member $R$ is composed of all responses that hash to
$H(R)$. As a consequence, any response $R \in C$ is representative of $C$.

To efficiently calculate a consensus, we rely on the property that the possible
consensus sets are disjoint. While iterating through the responses, we
keep track of the frequency of each hash, along with the maximum frequency
encountered and a response with a hash associated with that maximum frequency.
If the maximum frequency exceeds $T$ after going through all of the responses,
we return the response associated with that freqency. Note that this works
because we can choose any response in $C$ as a representative response.
If no consensus can be found or if the most frequent hash is \texttt{null}
(that is, failed responses), then an error response is returned to the user.

\begin{figure*}
  \centering
  \begin{adjustbox}{width=\textwidth}
  \begin{tikzpicture}[auto]
    \tikzset{every node}=[font=\tiny\sffamily]

    \node [block] at (0, 0) (client) {Client};
    \node [block] at (2, 0) (vhttp) {vHTTP};
    \node [block] at (4, 2) (vp-A) {Vantage Point A};
    \node [block] at (4, 1) (vp-B) {Vantage Point B};
    \node [block] at (4, 0) (vp-C) {Vantage Point C};
    \node [block] at (4, -1) (vp-D) {Vantage Point D};
    \node [block] at (4, -2) (vp-E) {Vantage Point E};
    \node [block] at (6, 0) (vhttp-rec) {vHTTP};
    \node [block] at (8, 0) (client-rec) {Client};

    \path[->, every
        node/.style={font=\sffamily\tiny,midway,anchor=center,fill=white}]
      (client) edge node {$R$} (vhttp)
      (vhttp) edge node {$R'$} (vp-A)
              edge node {$R'$} (vp-B)
              edge[dashed] node {$R'$} (vp-C)
              edge node {$R'$} (vp-D)
              edge node {$R'$} (vp-E)
      (vp-A) edge[ForestGreen] node {$S_A$} (vhttp-rec)
      (vp-B) edge[ForestGreen] node {$S_B$} (vhttp-rec)
      % (vp-C) edge[dashed] node {$S_C$} (vhttp-rec)
      (vp-D) edge[red] node {$S_D$} (vhttp-rec)
      (vp-E) edge[ForestGreen] node {$S_E$} (vhttp-rec)
      (vhttp-rec) edge[ForestGreen] node {$S_E$} (client-rec);

  \end{tikzpicture}
  \end{adjustbox}
  % TODO: expand caption
  \caption{
    \emph{Flow of data when vHTTP achieves consensus.}
    In this example, the client proxies a request $R$ through vHTTP. vHTTP
    sends $R$, with some minor modifications\protect\footnotemark,
    along to the five vantage points configured. Vantage points $A$, $B$, and
    $C$ respond. Assuming a default threshold of $0.5$, they achieve
    consensus, so one of those responses (in this case, we arbitrarily chose
    $S_E$) is sent back to the user. $D$ is either a malicious proxy or a MITM
    attack is being performed on this path, so its response is discarded as
    against the largest consensus, and $C$ never responds because it is
    unavailable.
    }
  \label{fig:vhttp-consensus-flow}
\end{figure*}

\footnotetext{
  The primary modification is to ensure that the
  request has the \texttt{Connection} header set to \texttt{close} so that
  the connection does not remain open
  }


\subsection{Implementation}

vHTTP is implemented in Python using \texttt{asyncio} and \texttt{aiohttp}.
We use \texttt{aiohttp} because it provides nice abstractions for working with
HTTP. More importantly, \texttt{aiohttp} allows for asynchronous request
handling and proxying, which means vHTTP can make use of the concurrency
opportunities available with network operations. In a synchronous mode, a
proxy that forwards each request to $N$ proxies would be terribly slow, as it
would take the combined time of each proxy's response to actually respond to
the original client. With asynchronous handling, a request to each proxy is
sent immediately and then all of the responses are gathered and checked for
consensus. Through this manner, the total time taken to acquire the proxy
responses is that of the slowest response, not the combined total.

Similarly, multiple simultaneous requests to vHTTP can be efficiently handled
because whenever the server is waiting for the vantage points to respond, it
can process another request. These asynchronous features work to minimize
the overhead of using vHTTP.
