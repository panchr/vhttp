\section{Introduction}
\label{sec:intro}

HTTP is incredibly widespread and at the same time, incredibly insecure. HTTP
sends data over plaintext and does not provide any mechanism to verify the
authenticity or integrity of data. As a consequence, the data received over
plain HTTP cannot be trusted.

With no security guarantees, numerous attacks are possible upon HTTP, including
Man-in-the-Middle (MITM) attacks. In MITM attacks, a third party intercepts,
and potentially forges, communication between two parties. With HTTP's
client-server model, a third-party can intercept requests from the client and
respond with bogus HTTP responses. Without any added security, the client has
no way to detect that such an attack is occurring and will accept the bogus
response as legitimate.

HTTPS alleviates most of these concerns through the use of certificates, which
render remote MITM attacks nearly impossible to perform. In addition, HTTPS
operates over TLS (Transport Layer Security) or SSL (Secure Sockets Layer) that
encrypts data from both the client and server. However, not all servers allow
or support HTTPS. Currently, about 21.6\% of Alexa's top 1,000,000 websites do
not support HTTPS \cite{Censys-HTTPS}.\footnote{
  The pervasiveness of HTTP exists even with higher Alexa ranks; 19.6\% of the
  top 10,000 websites and 11\% of the top 100 do not support HTTPS
  \cite{Censys-HTTPS}.
  }

Even if they do support HTTPS, websites often include
download links (or more generally, instructions on how to download static
content) that operate over HTTP. A simple solution to MITM attacks, then, is
to upgrade servers to HTTPS; however, this is not possible without cooperation
from the server administrators. A client wishing to validate the authenticity
and integrity of data from a server over HTTP cannot do so.

vHTTP is concerned with accessing static content over HTTP. Although most
static content, such as images or text files, are not that harmful if a third
party presents a bogus version, there are specific types of content
which can be devastating. For example, downloading programs or installation
scripts over HTTP, and then executing them, can have disastrous consequences
were a MITM attack to be performed. Unfortunately, the practice of presenting
HTTP links for this type of content is not uncommon, which presents a ripe
opportunity for malicious agents to perform MITM attacks.

The prevalence of such HTTP links for static content motivates the vHTTP
project, which aims to mitigate MITM attacks. This mitigation is accompished on
the client-side and so, can be performed without changes to the servers hosting
the content (which are generally not in the client's control).
