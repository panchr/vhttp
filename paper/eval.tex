\section{Evaluation}
\label{sec:eval}

We evaluate vHTTP on three primary metrics: effectiveness, deployability, and
performance.

% TODO: effectiveness and performance sections

\subsection{Effectiveness}

vHTTP's effectiveness is defined as its ability to mitigate different MITM
attacks. By design, it is able to mitigate localized MITM attacks. That is, it
can prevent the situation where a single path or a subset of the paths to the
destination are compromised. In the event where all or a majority of the paths
are under a MITM attack, such as if through a global-scale DNS-based attack or
if an attacker gets access to the servers themselves, vHTTP will not help.

With a threshold setting of $T$ and $N$ total vantage points, vHTTP can
mitigate an attack where $N-T$ vantage points are compromised; consequently,
it is vital that the user configures a diverse set of vantage points that are
unlikely to be compromised all together. Against small-scale attacks, vHTTP
has shown to be effective in tests. A more detailed example of using vHTTP,
and how it is effective against localized attacks, is included in the project
repository.

\subsection{Deployability}

A significant portion of the design considerations made for vHTTP involve the
feasibility to use it without significant configuration or infrastructure
changes required. vHTTP works with existing infrastructure, in that it uses
readily-available free (or private) HTTP proxies as vantage points. More
importantly, vHTTP can function without changes to the servers being contacted;
it operates entirely from the client-side. As a consequence, new infrastructure
or changes to existing infrastructure is not required. Security-conscious users
can make use of vHTTP as-is and gain extra confidence in the authenticity and
integrity of response contents.

\subsection{Performance}
