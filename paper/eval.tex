\section{Evaluation}
\label{sec:eval}

We evaluate vHTTP on three primary metrics: effectiveness, deployability, and
performance.

\subsection{Effectiveness}

vHTTP's effectiveness is defined as its ability to mitigate different MITM
attacks. By design, it is able to mitigate localized MITM attacks. That is, it
can prevent the situation where a single path or a subset of the paths to the
destination are compromised. In the event where all or a majority of the paths
are under a MITM attack, such as if through a global-scale DNS-based attack or
if an attacker gets access to the servers themselves, vHTTP will not help.

With a threshold setting of $T$ and $N$ total vantage points, vHTTP can
mitigate an attack where $N-T$ vantage points are compromised; consequently,
it is vital that the user configures a diverse set of vantage points that are
unlikely to be compromised all together. Against small-scale attacks, vHTTP
has shown to be effective in tests. A detailed example of using vHTTP, and how
it is effective against localized attacks, is included in the project
repository.

\subsection{Deployability}

A significant portion of the design considerations made for vHTTP involve the
feasibility to use it without significant configuration or infrastructure
changes required. vHTTP works with existing infrastructure, in that it uses
readily-available free (or private) HTTP proxies as vantage points. More
importantly, vHTTP can function without changes to the servers being contacted;
it operates entirely from the client-side. As a consequence, new infrastructure
or changes to existing infrastructure is not required. Security-conscious users
can make use of vHTTP as-is and gain extra confidence in the authenticity and
integrity of response contents.

\subsection{Performance}

Before looking at the performance of vHTTP, we note that it is not meant to be
incredibly performant, as it is not a general-purpose proxy. For one, it is
only oriented towards retrieving static content, over HTTP, such that multiple
requests \emph{should} have the same response. More importantly, we expect that
most of a security-conscious user's traffic is using secure protocols like
HTTPS, not HTTP. vHTTP is built to be used as a last resort when accessing
static content that cannot be accessed over HTTPS, which we hope is an
ever-decreasing amount of content. With that in mind, we briefly go into the
performance of vHTTP.

Because it is another proxy, vHTTP incurs some overhead just from the
additional hop; instead of simply sending a request to an endpoint, the user's
client sends a request to vHTTP, which will then send more requests out.
Figure \ref{figs:general-benchmark} shows the incurred overhead. Compared to
sending requests without a proxy, vHTTP has slightly greater fluctuation. This
is likely due to the fact that because vHTTP has more requests (to multiple
vantage points), there is more network variance. The overhead vHTTP faces is
not proportional to the initial request time; instead, it is simply a
relatively constant amount of overhead to vHTTP plus additional delays that
each vantage point may face.

vHTTP also does not scale too well with concurrent requests. Particularly,
after 30 concurrent requests, the time taken per request increases more so
than before, as seen in Figure \ref{figs:concurrency-benchmark}. We believe
this is due to the single-threaded nature of the proxy; even with asynchronous
handling, requests will be queued while others are being processed. Note that
more thorough performance data would have to be collected to get an accurate
comparison of vHTTP's performance versus normal performance. The metrics
collected so far are generally rough. That being said, we do not think the
performance under load or the incurred overhead is a major problem for
vHTTP. As mentioned before, the proxy is not designed to be extremely
performant; it is designed for use in specific use cases which we imagine have
a limited number of concurrent requests and can easily tolerate a small
overhead. Consequently, we did not pursue improving the performance too heavily
as this was not a major concern; vHTTP performs well enough to be usable for
its intended purposes.

\begin{figure}
  \begin{tikzpicture}
    \begin{axis}[
      title={vHTTP Request Timing},
      ylabel={Time (ms)},
      xlabel={Request Number},
      ymin=0, ymax=200,
      legend pos=north west,
      ymajorgrids=true,
      grid style=dashed]
    \addplot+[only marks, mark size = 1pt, mark options=ForestGreen]
      table[x=run, y=proxy, col sep=comma]
      {data/general.csv};
    \addlegendentry{vHTTP}

    \addplot+[only marks, mark size = 1pt, mark options=blue]
      table[x=run, y=no-proxy, col sep=comma]
      {data/general.csv};
    \addlegendentry{normal}
    \end{axis}
  \end{tikzpicture}

  \caption{
    General performance data of requests to various websites. vHTTP was
    configured with 4 vantage points.
    }
\end{figure}

\begin{figure}
  \begin{tikzpicture}
    \begin{axis}[
      title={vHTTP Scalability},
      ylabel={Time (ms)},
      xlabel={Concurrent Requests},
      ymin=0, ymax=1500,
      legend pos=north west,
      ymajorgrids=true,
      grid style=dashed]
    \addplot+[mark size = 1pt, ForestGreen, mark options=ForestGreen]
      table[x=concurrency, y=75th, col sep=comma]
      {data/aggregate-proxy.csv};
    \addlegendentry{vHTTP}

    \addplot+[mark size = 1pt, blue, mark options=blue]
      table[x=concurrency, y=75th, col sep=comma]
      {data/aggregate-no-proxy.csv};
    \addlegendentry{Normal}
    \end{axis}
  \end{tikzpicture}

  \caption{
    Performance as number of concurrent requests increases. vHTTP was
    configured with 3 vantage points.
    }
  \label{figs:concurrency-benchmark}
\end{figure}

